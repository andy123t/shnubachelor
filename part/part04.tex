
% 表格环境

\clearpage
\section{表格环境}

\subsection{表的使用}

LaTeX 的 Table 环境是一个浮动体环境, 排版与 Figure 环境类似. 作为论文, 推荐使用三线表进行排版. 一般的三线表, 标题前后有横线, 表格最后有横线. %当然也可以根据需要在合适的地方加线.

本文基于 tabularx 宏包定义了新的的左中右 (LCR) 格式, tabularx 环境需要先定义表格的总宽度, LCR 三个格式根据表格的总宽度自行控制列宽, 且其宽度相等. 本模板还定义了命令 \verb|P{}|, 它设置某一列宽度且内容居中 (如 \verb|P{1cm}| 控制某一列的宽度为 1cm), 实际上 \verb|P{}| 命令是在 \verb|p{}| 命令的基础上增加了居中功能.

\subsection{表格示例}

如下表格: 表~\ref{tab:heightweight}. 通过 \verb|autoref| 引用表格: \autoref{tab:heightweight}.

\begin{table}[!htp]
\centering
% PLCR已经定义
\caption{某校学生升高体重样本}
\label{tab:heightweight}
\begin{tabularx}{0.9\textwidth}{lCCC}
   \toprule
	序号 & 年龄 & 身高 & 体重 \\
	\midrule
	001 & 15 & 156 & 42 \\
	002 & 16 & 158 & 45 \\
	003 & 14 & 162 & 48 \\
	004 & 15 & 163 & 50 \\
    \cmidrule{2-4}
	平均 & 15 & 159.75 & 46.25 \\
	\bottomrule
\end{tabularx}
\end{table}

\begin{table}[htp!]
\centering
\caption{论文进度安排}
%\renewcommand\arraystretch{1.1} %定义表格高度
% PLCR前面已经定义
\begin{tabularx}{0.8\textwidth}{|P{4cm}|C|}
\Xhline{2\arrayrulewidth}
论文起止时间       &  论文筹备过程 \\
\hline
20xx.xx -- 20xx.xx    &  论文定题,整理相关文献 \\
\hline
20xx.xx -- 20xx.xx    &  审查、修改、完成开题报告 \\
\hline
20xx.xx -- 20xx.xx   &  对论文排版、初步完成论文初稿 \\
\hline
20xx.xx -- 20xx.xx    &  毕业论文预答辩 \\
\hline
20xx.xx -- 20xx.xx    &  对论文进行补充、完善 \\
\hline
20xx.xx -- 20xx.xx    &  论文定稿 \\
\hline
20xx.xx -- 20xx.xx    &  毕业论文答辩 \\
\Xhline{2\arrayrulewidth}
\end{tabularx}
\end{table}


\clearpage
基于 tabular 环境设置一些格式: 上下表格线加粗, 如表~\ref{tab:error1}.
\begin{table}[!htp]
%\small
\centering
%\renewcommand\arraystretch{1.1} %定义表格高度
\caption{数值误差与收敛速率示例}
\label{tab:error1}
\begin{tabular}{c|c|cc|cc|cc}
\Xhline{2\arrayrulewidth}
degree &  step-size~$h$  & $L^2$-errors  &  order  & $H^1$-errors & order & $L^\infty$-errors  &  order \\
\hline
  & 1/128 & 9.18E-06 & 2.02 & 7.70E-03 & 1.01  & 6.46E-07 & 2.02 \\
1 & 1/256 & 2.29E-06 & 2.01 & 1.92E-03 & 1.00  & 1.61E-07 & 2.01 \\
  & 1/512 & 5.70E-07 & 2.00 & 9.56E-04 & 1.00  & 4.01E-08 & 2.00 \\
\hline
  & 1/128 & 1.39E-08 & 3.01 & 1.15E-05 & 2.01  & 3.48E-12 & 4.02 \\
2 & 1/256 & 1.73E-09 & 3.01 & 2.88E-06 & 2.01  & 3.27E-13 & 3.94 \\
  & 1/512 & 2.17E-10 & 3.00 & 7.24E-06 & 2.00  & 6.66E-13 & 1.55 \\
\hline
  & 1/32  & 2.28E-09 & 4.05 & 6.92E-07 & 3.04  & 1.45E-15 & 8.21 \\
3 & 1/64  & 1.42E-10 & 4.03 & 8.65E-08 & 3.02  & 2.06E-14 & 3.85 \\
  & 1/128 & 8.91E-12 & 4.01 & 1.08E-08 & 3.01  & 3.86E-14 & 0.91 \\
\Xhline{2\arrayrulewidth}
\end{tabular}
\end{table}

基于 tabularx 环境设置一些格式, 如表~\ref{tab:error2}.
\begin{table}[htp!]
\centering
%\renewcommand\arraystretch{1.1} %定义表格高度
\caption{数值误差示例}
\label{tab:error2}
\begin{tabularx}{0.96\textwidth}{|P{0.8cm}|C|C|C|C|C|C|}
\Xhline{2\arrayrulewidth}
N  & A       & B    & C       & D      & E       & F   \\
\Xhline{2\arrayrulewidth}
2  & 9.20E-05 & 9.90E-05 & 1.00E-06 & 8.00E-06 & 1.50E-05 & 6.70E-05 \\
4  & 9.80E-05 & 8.00E-05 & 7.00E-06 & 1.40E-05 & 1.60E-05 & 7.30E-05 \\
6  & 4.00E-06 & 8.10E-05 & 8.80E-05 & 2.00E-05 & 2.20E-05 & 5.40E-05 \\
8  & 8.50E-05 & 8.70E-05 & 1.90E-05 & 2.10E-05 & 3.00E-06 & 6.00E-05 \\
10 & 8.60E-05 & 9.30E-05 & 2.50E-05 & 2.00E-06 & 9.00E-06 & 6.10E-05 \\
12 & 1.70E-05 & 2.40E-05 & 7.60E-05 & 8.30E-05 & 9.00E-05 & 4.20E-05 \\
14 & 2.30E-05 & 5.00E-06 & 8.20E-05 & 8.90E-05 & 9.10E-05 & 4.80E-05 \\
16 & 7.90E-05 & 6.00E-06 & 1.30E-05 & 9.50E-05 & 9.70E-05 & 2.90E-05 \\
18 & 1.00E-05 & 1.20E-05 & 9.40E-05 & 9.60E-05 & 7.80E-05 & 3.50E-05 \\
20 & 1.10E-05 & 1.80E-05 & 1.10E-04 & 7.70E-05 & 8.40E-05 & 3.60E-05 \\
\Xhline{2\arrayrulewidth}
\end{tabularx}
\end{table}

